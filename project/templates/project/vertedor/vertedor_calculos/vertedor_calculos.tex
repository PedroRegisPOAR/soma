\documentclass{article}
\usepackage[a4paper, left=3cm, right=3cm, top=2cm, bottom=2cm]{geometry}
\usepackage[brazil]{babel}
%\usepackage[utf8]{inputenc} % Por alguma razão diabólica estava dando problema com os encodes
\usepackage[T1]{fontenc}

\usepackage{parskip} % Para remover identação desnecessaria.

\usepackage{amsmath} % Para poder usar \implies

\usepackage{lipsum}

\newcommand{\myspace}{0.2cm}

\begin{document}
\pagenumbering{gobble}

Dados: $\mathrm{Q = {{Q}} \ m^{3}/s}$, $\mathrm{T = {{T}} ^{\circ}C}$, $\mathrm{g = {{g}} \ m/s^{2}}$, $\mathrm{ \gamma = {{gamma}} \ N/m^{3}}$, $\mathrm{ \mu = {{mu}} \ N \! \cdot \! s \cdot m^{-2}}$, $\mathrm{P_{vr} = {{Pvr}} \ m }$, $\mathrm{ b = {{b}} \ m }$.  \\

1. Altura, $\mathrm{ H_{vr} }$.
\vspace{\myspace}

\begin{center}
	$ 
		\mathrm
		{ 
			H_{vr} = \left ( \dfrac{ Q }{ 1.848b } \right )^{ \frac{3}{2} }
		} 
		\implies
		\mathrm
		{
			H_{vr} = \left ( \dfrac{ {{Q}} m^{3}/s}{ 1.848 \cdot {{b}} \ m } \right )^{ \frac{3}{2} }
		} 
		\implies
		\fbox
		{ 
			$\mathrm
			{ 
				H_{vr} = {{Hvr}} \ m
			}$
		} 
	$ 
\end{center}
\vspace{\myspace}

2. Distância entre vertedor e seção inicial do ressalto $\mathrm{ L_{m} }$, profundidade crítica $\mathrm{ y_{c} }$, e altura da lâmina d'água no início do ressalto $\mathrm{ y_{1} }$.
\vspace{\myspace}
\begin{center}
	$ 
		\mathrm
		{ 
			y_{c}=\sqrt[3]{ \dfrac{ \mathrm{Q}^{2} }{ \mathrm{g}\mathrm{b}^2} \ }
		} 
		\implies
		\mathrm
		{
			y_{c} = \sqrt[3]{ \dfrac{ {{Q}}^{2} \left( \mathrm{m^{3}/s} \right )^{2} }{ {{g}} \mathrm{m/s^{2}} \ {{b}}^{2} \mathrm{m^{2}} } }
		} 
		\implies
		\fbox
		{ 
			$\mathrm
			{ 
				y_{c} = {{yc}} \ m
			}$
		} 
	$ 
\end{center}
\vspace{\myspace}

\begin{center}
	$ 
		\mathrm
		{ 
			L_{m} = H_{vr}\left ( \dfrac{P_{vr}}{H_{vr}} \right )^{0.54}
		} 
		\implies
		\mathrm
		{
			L_{m} = {{Hvr}}m\left ( \dfrac{ {{Pvr}}m }{ {{Hvr}}m } \right )^{0.54}
		} 
		\implies
		\fbox
		{ 
			$\mathrm
			{ 
				L_{m} = {{Lm}} \ m
			}$
		} 
	$ 
\end{center}
\vspace{\myspace}

\begin{center}
	$ 
		\mathrm
		{ 
			y_{1} = \dfrac{ 1.414y_{c} }{ \sqrt{ 2.56 + \dfrac{ P_{vr} }{ y_{c} } } }
		} 
		\implies
		\mathrm
		{
			y_{1} = \dfrac{1.414 \cdot {{yc}}m}{ \sqrt{ 2.56 + \dfrac{ {{Pvr}}m }{ {{yc}}m } } }
		} 
		\implies
		\fbox
		{ 
			$\mathrm
			{
				y_{1}= {{y1}} \ m
			}$
		} 
	$ 
\end{center}
\vspace{\myspace}

3. Velocidade do escoamento no inicio do ressalto, $\mathrm{ v_{1} }$.
\vspace{\myspace}
\begin{center}
	$ 
		\mathrm
		{ 
			v_{1} = \dfrac{ Q }{ y_{1}b }
		} 
		\implies
		\mathrm
		{
			v_{1} = \dfrac{ {{Q}} m^{3}/s}{ {{y1}}m \  {{b}}m }
		} 
		\implies
		\fbox
		{ 
			$\mathrm{
				v_{1} = {{v1}} \ m/s
			}$
		} 
	$ 
\end{center}
\vspace{\myspace}

4. Número de Froude na seção 1, $\mathrm{ F_{1} }$.
\vspace{\myspace}
\begin{center}
	$ 
		\mathrm
		{ 
			F_{1} = \dfrac{ v_{1} }{ \sqrt{ gy_{1} } }
		} 
		\implies
		\mathrm
		{
			F_{1} = \dfrac{ {{v1}} m/s }{ \sqrt{ {{g}} m/s^{2} \ {{y1}}m } }
		} 
		\implies
		\fbox
		{ 
			$\mathrm
			{ 
				F_{1} = {{F1c}}
			}$
		} 
	$ 
\end{center}
\vspace{\myspace}

5. Altura da lâmina d'água no final do ressalto $\mathrm{ y_{2} }$.
\vspace{\myspace}
\begin{center}
	$
		\mathrm
		{
			y_{2} = \dfrac{ y_{1} }{2}\left ( \sqrt{ 1+ 8F_{1}^{2} } - 1\right )
		}
		\implies 
		\mathrm
		{
			y_{2} = \dfrac{ {{y1}}m }{2} \left ( \sqrt{ 1+ 8 \cdot {{F1}} ^{2} } - 1 \right )
		}
		\implies 
		\fbox
		{
			$\mathrm
			{
				y_{2} = {{y2}} \ m
			}$
		}
	$  
\end{center}
\vspace{\myspace}

6. Perda de energia $\mathrm{ E_{n} }$, comprimento do ressalto $\mathrm{ L_{r} }$, e velocidade no final do ressalto $\mathrm{ v_{2} }$.

\vspace{\myspace}
\begin{center}
	$
		\mathrm
		{
			E_{n} = \dfrac{ \left ( y_{2} - y_{1} \right)^{3} }{ 4y_{1}y_{2} } 
		} 
		\implies
		\mathrm
		{
			E_{n} = \dfrac{ \left ( {{y2}}m - {{y1}}m \right)^{3} }{ 4\cdot {{y1}}m \ {{y2}}m }
		}
		\implies 
		\fbox
		{
			$\mathrm
			{
				E_{n} = {{En}} \ m
			}$
		}
	$  
\end{center}
\vspace{\myspace}


\begin{center}
	$ 
		\mathrm
		{ 
			v_{2} = \dfrac{ Q }{ y_{2}b }
		} 
		\implies
		\mathrm
		{
			v_{2} = \dfrac{ {{Q}} m^{3}/s}{ {{y2}}m \ {{b}}m }
		} 
		\implies
		\fbox
		{ 
			$\mathrm
			{
				v_{2} = {{v2}} \ m/s
			}$
		} 
	$ 
\end{center}
\vspace{\myspace}


\begin{center}
	$
		\mathrm
		{
			L_{r} = c\left ( y_{2} - y_{1} \right)
		} 
		\implies
		\mathrm
		{
			L_{r} = {{c}}\left ( {{y2}}m - {{y1}}m \right)
		}
		\implies 
		\fbox
		{ 
			$\mathrm
			{
				L_{r}= {{Lr}} \ m
			}$
		}
	$  
\end{center}
\vspace{\myspace}


7. Velocidade média, $\mathrm{ U_{m} }$.
\begin{center}
	$ 
		\mathrm
		{ 
			U_{m} = \dfrac{ v_{1} + v_{2} }{2}
		} 
		\implies
		\mathrm
		{
			U_{m} = \dfrac{ {{v1}}m/s + {{v2}}m/s }{2}
		} 
		\implies
		\fbox
		{ 
			$\mathrm{ U_{m} = {{Um}} \ m/s}$
		} 
	$ 
\end{center}
\vspace{\myspace}

8. Tempo médio de mistura, $\mathrm{ T_{m} }$.
\vspace{\myspace}
\begin{center}
	$
		\mathrm
		{
			T_{m} = \dfrac{ L_{r} }{ U_{m} } 
		} 
		\implies
		\mathrm
		{
			T_{m} = \dfrac{ {{Lr}}m }{ {{Um}}m/s } 
		}
		\implies 
		\fbox
		{ 
			$\mathrm
			{
				T_{m}= {{Tm}} \ s
			}$
		}
	$  
\end{center}
\vspace{\myspace}

9. Gradiente de velocidade médio, $\mathrm{ G_{m} }$.
\vspace{\myspace}
\begin{center}
	$
		\mathrm
		{
			Gm = \sqrt{ \dfrac{ \gamma E_{n} }{ \mu T_{m} } }
		} 
		\implies
		\mathrm
		{
			Gm = \sqrt{\dfrac{ {{gamma}} N/m^{3} \cdot {{En}}m }{ {{mu}}N \! \cdot \! s \cdot m^{-2} \ {{Tm}}s } }
		}
		\implies 
		\fbox
		{ 
			$\mathrm
			{
				G_{m} = {{Gm}} \ s^{-1}
			}$
		}
	$  
\end{center}

\end{document} 