\documentclass{article}
\usepackage[a4paper, left=3cm, right=2cm, top=2cm, bottom=2cm]{geometry}
\usepackage[brazil]{babel}
\usepackage[utf8]{inputenc} % Por alguma razão diabólica estava dando problema com os encodes
\usepackage[T1]{fontenc}
\usepackage{parskip} % Para remover identação desnecessaria.
\usepackage{amsmath} % Para poder usar \implies

\newcommand{\myspace}{0.3cm}

\begin{document}
\pagenumbering{gobble}

Dados: $\mathrm{Q_{max} = {{Qmáx}} \ m^{3}/s}$, $\mathrm{Q_{med} = {{Qméd}} \ m^{3}/s}$, $\mathrm{Q_{min} = {{Qmín}} \ m^{3}/s}$, espessura da grade $\mathrm{t = {{t}} \ mm}$,
espaçamento da grade $\mathrm{a = {{a}} \ mm}$, templo de limpeza $\mathrm{tl = {{tl}} \ dia}$, aceleração da gravidade, $\mathrm{g = {{g}} \ m/s^{2}}$, areia acumulada $\mathrm{Areia_{Acu} = {{AreiaAcu}} \ \frac{L}{1000m^{3}}}$, velocidade horizontal do fluxo na caixa de areia $\mathrm{V = {{V}} \ m/s}$, velocidade nas passagens da grade $\mathrm{V_{g} = {{Vg}} \ m/s}$, relação entre o comprimento L e a altura H da caixa de areia $\mathrm{c = {{c}}}$. Escolhendo $\mathrm{W = {{W}} \ cm }$, consequentemente $\mathrm{n = {{n}} }$ e $\mathrm{ \lambda = {{lamb}} }$.


Dimensionamento calha parshall:

\vspace{\myspace}

Altura máxima:

\vspace{\myspace}

\begin{center}
	$
		\mathrm
		{
			H_{max} = \sqrt[n]{\dfrac{ \mathrm{ Q_{max}} }{\lambda}}
		} 
		\implies
		\mathrm
		{
			H_{max} = \sqrt[ {{n}} ]{ \dfrac{ {{Qmáx}} \mathrm{m^{3}/s} }{ {{lamb}} } }
		} 
		\implies 
		\fbox
		{
			$\mathrm
			{
				H_{max} = {{Hmáx}} \ m
			}$
		}
	$
\end{center}

\vspace{\myspace}

Altura mínima:

\vspace{\myspace}

\begin{center}
	$
		\mathrm
		{
			H_{min} = \sqrt[n]{\dfrac{ \mathrm{ Q_{min} } }{\lambda}}
		} 
		\implies
		\mathrm
		{
			H_{min} = \sqrt[ {{n}} ]{ \dfrac{ {{Qmín}} \mathrm{m^{3}/s} }{ {{lamb}} } }
		} 
		\implies 
		\fbox
		{
			$\mathrm
			{
				H_{min} = {{Hmín}} \ m
			}$
		}
	$
\end{center}


Cálculo do ressalto:

\vspace{\myspace}


\begin{center}
	$
		\mathrm
		{
			Z  = \dfrac{H_{min}Q_{max} - Q_{min}H_{min}}{Q_{max} - Q_{min}}
		} 
		\implies
		\mathrm
		{
			Z  = \dfrac{ {{Hmín}}m \ {{Qmáx}}m^{3}/s - {{Qmín}}m^{3}/s \ {{Hmín}} }{ {{Qmáx}} m^{3}/s - {{Qmín}}m^{3}/s }
		} 
		\implies 
		\fbox
		{
			$\mathrm
			{
				Z = {{Z}} \ m
			}$
		}
	$
\end{center}


\vspace{\myspace}

Nível mínimo da água antes do ressalto da calha parshall:

\begin{center}
	$
		\mathrm
		{
			y_{min} = H_{min} - Z
		} 
		\implies
		\mathrm
		{
			y_{min} = {{Hmín}}m - {{Z}}m
		} 
		\implies 
		\fbox
		{
			$\mathrm
			{
				y_{min} = {{ymín}} \ m
			}$
		}
	$
\end{center}

\vspace{\myspace}


Nível máximo da água antes do ressalto da calha parshall:

\vspace{\myspace}

\begin{center}
	$
		\mathrm
		{
			y_{max} = H_{max} - Z
		} 
		\implies
		\mathrm
		{
			y_{max} = {{Hmáx}}m - {{Z}}m
		} 
		\implies 
		\fbox
		{
			$\mathrm
			{
				y_{max} = {{ymáx}} \ m
			}$
		}
	$
\end{center}
\vspace{\myspace}




Cálculo da grade:

\vspace{\myspace}

Eficiência:

\vspace{\myspace}

\begin{center}
	$
		\mathrm
		{
			E = \dfrac{a}{a + t}
		} 
		\implies
		\mathrm
		{
			E = \dfrac{ {{a}} mm }{ {{a}} mm + {{t}} mm }
		} 
		\implies 
		\fbox
		{
			$\mathrm
			{
				E = {{E}}
			}$
		}
	$
\end{center}

\vspace{\myspace}

Área útil:

\begin{center}
	$
		\mathrm
		{
			A_{u} = \dfrac{Q_{max}}{V_{g}}
		} 
		\implies
		\mathrm
		{
			A_{u} = \dfrac{ {{Qmáx}}m^{3}/s }{ {{Vg}} m/s}
		} 
		\implies 
		\fbox
		{
			$\mathrm
			{
				A_{u} = {{Au}} \ m^{2}
			}$
		}
	$
\end{center}

\vspace{\myspace}

Área da seção do canal:

\vspace{\myspace}

\begin{center}
	$
		\mathrm
		{
			S = \dfrac{A_{u}}{E}
		} 
		\implies
		\mathrm
		{
			S = \dfrac{ {{Au}} m^{2}}{ {{E}} }
		} 
		\implies 
		\fbox
		{
			$\mathrm
			{
				S = {{S}} \ m^{2}
			}$
		}
	$
\end{center}

\vspace{\myspace}

Largura do canal da grade:

\vspace{\myspace}

\begin{center}
	$
		\mathrm
		{
			b = \dfrac{S}{H_{max} - Z}
		} 
		\implies
		\mathrm
		{
			b = \dfrac{ {{S}}m^{2} }{ {{Hmáx}}m - {{Z}}m }
		} 
		\implies 
		\fbox
		{
			$\mathrm
			{
				b = {{b}} \ m
			}$
		}
	$
\end{center}
\vspace{\myspace}

\newpage

Perda de carga na grade:
\vspace{\myspace}

Limpa:
\vspace{\myspace}

\begin{center}
	$
		\mathrm
		{
			\Delta h_{L} = 1.43 \dfrac{  V_{g}^{2} - V_{0}^{2} }{2g}
		} 
		\implies
		\mathrm
		{
			\Delta h_{L} = 1.43 \dfrac{ \left ( {{Vg}} m/s \right )^{2} - \left ( {{V0}} m/s\right )^{2} }{ 2\cdot {{g}} m/s^{2} }
		} 
		\implies 
		\fbox
		{
			$\mathrm
			{
				\Delta h_{L} = {{Deltah_limpa}} \ m
			}$
		}
	$
\end{center}



Com 50\% obstruida:
\vspace{\myspace}

\begin{center}
	$
		\mathrm
		{
			\Delta h_{50\%} = 1.43 \dfrac{ \left ( 2 V_{g} \right )^{2} - V_{0}^{2} }{2g}
		} 
		\implies
		\mathrm
		{
			\Delta h_{50\%} = 1.43 \dfrac{ \left ( 2 \cdot  {{Vg}} m/s \right )^{2} - \left ( 2 \cdot {{V0}} m/s \right )^{2} }{ 2\cdot {{g}} m/s^{2} }
		} 
		\implies 
		\fbox
		{
			$\mathrm
			{
				\Delta h_{50\%} = {{Deltah_50}} \ m
			}$
		}
	$
\end{center}
\vspace{\myspace}


Caixa de areia (desarenador):
\vspace{\myspace}

Área da seção transversal:
\vspace{\myspace}

\begin{center}
	$
		\mathrm
		{
			A = \dfrac{Q_{max}}{V}
		} 
		\implies
		\mathrm
		{
			A = \dfrac{ {{Qmáx}} m^{3}/s}{ {{V}} m/s}
		} 
		\implies 
		\fbox
		{
			$\mathrm
			{
				A = {{A}} \ m^{2}
			}$
		}
	$
\end{center}
\vspace{\myspace}

Largura da caixa de areia:
\vspace{\myspace}

\begin{center}
	$
		\mathrm
		{
			B = \dfrac{A}{H_{max} - Z}
		} 
		\implies
		\mathrm
		{
			B = \dfrac{ {{A}}m^{2} }{ {{Hmáx}}m - {{Z}}m }
		} 
		\implies 
		\fbox
		{
			$\mathrm
			{
				B = {{B}} \ m
			}$
		}
	$
\end{center}
\vspace{\myspace}

Cálculo do comprimento: Usando que $ \mathrm{ H = H_{max} - Z}$:
\vspace{\myspace}

\begin{center}
	$
		\mathrm
		{
			L = c\left ( H_{max} - Z \right )
		} 
		\implies
		\mathrm
		{
			L = {{c}}\left ( {{Hmáx}}m - {{Z}}m \right )
		} 
		\implies 
		\fbox
		{
			$\mathrm
			{
				L = {{L}} \ m
			}$
		}
	$
\end{center}
\vspace{\myspace}

Taxa de escoamento superficial(TES):
\vspace{\myspace}

\begin{center}
	$
		\mathrm
		{
			TES_{max} = \dfrac{Q_{max}}{LB}
		} 
		\implies
		\mathrm
		{
			TES_{max} = \dfrac{ {{Qmáx}} m^{3}/s }{ {{L}}m \  {{B}}m } \dfrac{60\cdot 60 \cdot 24 \ s }{1dia}
		} 
		\implies 
		\fbox
		{
			$\mathrm
			{
				TES_{max} =  {{TESmáx}} \ \dfrac{m^{3}}{m^{2}dia}
			}$
		}
	$
\end{center}
\vspace{\myspace}

Cálculo do rebaixo:
\vspace{\myspace}

Volume médio de areia:
\vspace{\myspace}
\begin{center}
	$
		\mathrm
		{
			Q_{medAreia} = Areia_{Acu}Q_{med}
		} 
		\implies
		\mathrm
		{
			Q_{medAreia} = {{AreiaAcu}} \dfrac{L}{1000m^{3}}\cdot \dfrac{m^{3}}{10^{3}L} \ {{Qméd}} \dfrac{m^{3}}{s} \cdot \dfrac{60\cdot 60 \cdot 24 \ s }{1dia} 
		}
		\newline 
		\newline
		\newline
		\implies 
		\fbox
		{
			$\mathrm
			{
				Q_{medAreia} = {{QmédAreia}} m^{3}/dia
			}$
		}
	$
\end{center}
\vspace{\myspace}

O volume para limpeza a cada {{tl}} dias, é:
\vspace{\myspace}

\begin{center}
	$
		\mathrm
		{
			Vol_{medAreia} = tlQ_{medAreia}
		} 
		\implies
		\mathrm
		{
			Vol_{medAreia} = {{tl}}dia \ {{QmédAreia}} m^{3}/dia
		}
		\implies 
		\fbox
		{
			$\mathrm
			{
				Vol_{medAreia} = {{VmedAreia}} m^{3}
			}$
		}
	$
\end{center}
\vspace{\myspace}

Área em planta:
\vspace{\myspace}

\begin{center}
	$
		\mathrm
		{
			A_{p} = LB
		} 
		\implies
		\mathrm
		{
			A_{p} = {{L}}m \ {{B}}m
		}
		\implies 
		\fbox
		{
			$\mathrm
			{
				A_{p} = {{A_P}} m^{2}
			}$
		}
	$
\end{center}
\vspace{\myspace}

Altura da areia acumulada:
\vspace{\myspace}

\begin{center}
	$
		\mathrm
		{
			H_{AreiaAc} = \dfrac{Vol_{medAreia}}{A_{p}}
		} 
		\implies
		\mathrm
		{
			H_{AreiaAc} = \dfrac{ {{VmédAreia}} m^{3} }{ {{A_P}} m^2 }
		}
		\implies 
		\fbox
		{
			$\mathrm
			{
				H_{AreiaAc} = {{H_AreiaAc}} m
			}$
		}
	$
\end{center}


\end{document} 